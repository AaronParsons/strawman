
\documentclass[12pt,preprint]{aastex}
%\usepackage{geometry} % see geometry.pdf on how to lay out the page. There's lots.
%\geometry{a4paper} % or letter or a5paper or ... etc
% \geometry{landscape} % rotated page geometry
\usepackage{booktabs}
\usepackage{amsmath}
%\usepackage{topcapt}
% See the ``Article customise'' template for come common customisations
\newcommand{\MHz}{\textrm{MHz}}
\newcommand{\M}{\textrm{min}}
\newcommand{\Hr}{\textrm{h}}
\title{Imaging sensitivity estimates for HERA}
\author{Danny Jacobs }
\date{\centering \today } % delete this line to display the current date
%\slugcomment{\today}
%%% BEGIN DOCUMENT
\begin{document}
Simulations of the largest reionization structures predict amplitudes in the 10mK range on sub-degree/100Mpc scales (Mcquinn et al, 2007, ApJ 654).   The full HERA array, after application of a conservative foreground filter, will have a surface brightness sensitivity of less than 1mK, giving it ample sensitivity to detect HI emission.  
\begin{figure}[htb]
\includegraphics[width=0.75\textwidth]{HERA576_SNR_z=8.png}
\caption{A full instrumental simulation of a large scale reionization model (McQuinn 2007). Injected noise levels are equivalent to 100 hours of on-field observing and a conservative filtering of foregrounds operating above the ``horizon limited wedge''. Contours indicate regions with SNR>10. }
\end{figure}



\emph{Foreground limited sensitivity}\\
As discussed elsewhere, the sta.ge 1 sensitivity of 21cm observations is limited to observing outside of the ``foreground wedge".  In practice this limits the effective bandwidth over which coherent imaging will be foreground-free to around 1MHz, a redshift range of 0.005.
\end{document}