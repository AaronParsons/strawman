\documentclass[11pt]{article}
\usepackage{fullpage}

%Facilities, Equipment, and Other Resources: In order for NSF, and its
%reviewers, to assess the scope of a proposed project, all organizational
%resources necessary for, and available to a project, must be described in this
%section of the proposal (see GPG Chapter II.C.2.i for further information).
%Proposers should describe only those resources that are directly applicable.
%Proposers should include an aggregated description of the internal and external
%resources (both physical and personnel) that the organization and its
%collaborators will provide to the project, should it be funded. Such
%information must be provided in this section, in lieu of other parts of the
%proposal (e.g., budget justification, project description). The description
%should be narrative in nature and must not include any quantifiable financial
%information.

\begin{document}
\pagestyle{empty}

\section*{Facilities, Equipment, Other}

The work described in this proposal will take place at XXX locations: 

\subsection*{\it Arizona State University}

Several ASU low frequency developmental efforts will contribute directly to the HERA effort. ASU will support the deployment of an antenna beam mapping octocopter system which is currently undergoing testing for MWA, PAPER and other low frequency experiments. This system will enable the precise mapping of the antenna response as a function of frequency. Several developments from the ongoing Experiment to Detect the Global EoR Step (EDGES, supported by NSF grant \#1207761) will also be directly applicable.  Most notable is an ongoing effort to improve the precision of low frequency antenna response calibration by two orders of magnitude. The resources at the disposal of the ASU low frequency lab include a 5000 square foot dedicated electronics lab complete with vector network analyzers, FPGA development stations, high accuracy noise sources, outdoor antenna test facility, 6 axis micromill and several highly qualified technicians.

Adjacent to the Low Frequency lab is the Laboratory for Astronomical and Space Instrumentation (LASI) which assembles and tests space instrumentation. Proximity to the ASU Machine Shop (which has produced space-qualified hardware for Mars missions) offers quick custom machining and integration of pieces in instrument assembly. 

The ASU High Performance Computing Initiative maintains a 5000-node cluster (Saguaro) with 11 TB of aggregate RAM and 215 TB of high-speed disk space that is available to all students and researchers through an internal time allocation process. A cluster dedicated to low frequency astronomy, consisting of three high power (16 core, 128G of RAM, 75TB of storage), suitable for algorithm development and data exchange, is maintained by the co-I and available to all collaborators.



\subsection*{\it Massachussetts Institute of Technology}

As part of the MIT's group participation in the MWA and with support from an MRI grant, a dedicated cluster of PC-Linux computers and storage are being purchased and installed in a phased manner as they are needed to accommodate MWA data. The final configuration will consist of ~15 computers and >1.0 Pbytes of storage space. The computers will typically have 12 cores and 100 GB of ram, providing enough compute power to produce the data "cubes" required for EoR science as the data arrive at MIT. The MWA data-taking will be complete by May 2015, and the cluster processing capacity will be made available to HERA in September 2015, when it is anticipated that the standard pipeline MWA data processing will be complete.

At MIT, we have developed a 21 cm power spectrum estimation pipeline using quadratic estimator and Fisher matrix formalism to optimally measure power spectra and rigorously keep track of estimator errors and correlations.  The technique was adapted (Liu \& Tegmark 2011) from earlier work on the CMB and galaxy surveys, acclerated to run in $\mathcal{O}(N\log N)$ by exploting various symmetries (Dillon et al. 2013a), and applied successfully to MWA prototype data (Dillon et al. 2013b). The statistical technology and code is straightforwardly adaptable to analyzing HERA data.

The hexagonal layout of the HERA array core allows unprecedented opportunities for automated precision calibration and quality control using massive baseline redundancy (Liu et al 2010). This approach was successfully prototyped in the MITEoR experiment, and the resulting hardware and software is now available for HERA use. This includes the MITEoR 64 dual-polarization correlator, including 4 ADCx64 boards, 4 ROACH-II processing boards, 8 ROACH-I processing boards and two data acquisition servers, as well as 64 dual-polarization MITEoR elements with analog signal chains, including nodes, Walsh modulation boards, and receivers.

The massive baseline redundancy of HERA also provides a potential for significant efficiency improvements to the correlator though 2D FFT's on the hexagonal antenna lattice, cutting the numerical load scaling of the X-engine from $n^2$ to $n \log n$. This would free up a significant fraction of the GPU resources, which could instead be used for automated real-time calibration and quality control.


\subsection*{\it NRAO} 

Bradley support
Carilli support
Green Bank site
Infrastructure and systems associated with the PAPER-32 array deployed in Green Bank.
CST software for simulating feed responses

% XXX update

Our 500 sq. ft. field station is located on the NRAO Green Bank, WV observatory
site. It houses a work area and plenty of space to deploy instrumentation.
Positioned within the National Radio Quiet Zone, this site is an ideal venue
for very sensitive radio frequency measurements due to its remoteness from
large urban areas and its exceptional laboratory infrastructure. A relatively
large amount of outdoor space is available adjacent to the station for the
deployment of antennas under test. The satellite downlink measurement system is
here, and the PAPER array is located nearby. Electromagnetic enclosures are
used to house all instrumentation to prevent self-interference as well as
interference with Observatory telescopes, in strict adherence to the radio
emissions control policy of the Observatory’s Interference Protection Group
(IPG).

There warehouse facilities located in Green Bank are ideal for staging our
shipments to South Africa. Wooden crates are built in the carpenter shop and
moved to the warehouse for packing and short term storage. Heavy lift
equipment and the expertise to use them safely are also available to transport
these crates to the truck mounted shipping containers destined for the cargo
ship.

Our 800 sq.ft. research laboratory is located at the NRAO’s Technology Center.
It is equipped with two complete micro-assembly workstations for component
fabrication and rework as well as a suite of basic instruments necessary for
evaluation and repair. In addition, the laboratory contains several signal
generators, power meters, spectrum analyzers, an impedance bridge, network
analyzers, and noise figure meters for radio frequency applications. A small
anechoic chamber is available for antenna impedance measurements, and an
computer-controlled environmental chamber is ready for studies of thermal and
humidity induced effects on component and subsystem designs. We routinely make
use of the NRAO machine shop facilities.

We also have access to the latest computer-aided design software. Autocad
Inventor is used for three-dimensional mechanical drawings. Agilent’s Advanced
Design System (ADS) is available for circuit and subsystem modeling while CST
Microwave Studio is used for electromagnetic simulations of antenna and other
RF structures. In addition, CF Design is available for modeling thermal
conduction and radiation pathways.

\subsection*{\it PAPER}

PAPER 128 dual-polarization correlator, including 16 ADCx16 boards, 
8 ROACH-II processing boards, 8 servers each hosting 2 GTX690 dual-GPU cards,
and two data acquisition servers.
Data compression and storage system, including 5 computing servers and 
a XXX TB raid storage system.
128 PAPER elements, with analog signal chains, including nodes, receivers.
EMC enclosure housing the correlator and electronics.
Power infrastructure (transformers, UPS)
HVAC system.

\subsection*{\it South Africa}

Site clearing
Use of KAPB
Power + internet service
power + internet infrastructure
SKASA liason
accommodations
occasional operations maintenence
access roads to site
Bernardi support

% XXX update

PAPER’s been given the use of a 300m diameter cleared, level, circular area
located within the protection of South Africa’s Astronomy Geographic Advantage
Act. A 120dB RFI shielded 12m container has been placed in the center to house
electronics. The entire system is powered by 20kVA diesel generator though
installation of a grid connection is under investigation. The container houses
work space and two 19" 42U racks for PAPER systems. It is cooled by a 5 kW
capacity chiller plant. This should provide sufficient capacity for up to a 128
antennas.

In addition, SKA-SA assists with the unloading, assembly and storage of
shipping crates. Two on-site mobile truck cranes are available along with use
of a 30m x 18m dish construction shed, shared with the KAT-7 construction.
Accommodation is provided for on-site PAPER visitors and off-road pickup trucks
are available for use to, from and while on site.

A digital electronic laboratory is available in the Cape Town SKA-SA offices,
complete with hot-air rework stations and high frequency test equipment
(bench power supplies, signal generators, network analyzers, spectrum
analyzers, oscilloscopes up to 100GHz etc). There is an off-site RFI
measurement facility near Cape Town (Houwteq) able to perform measurements to,
amongst others, MIL-STD-461E (18GHz) with a greater than 100dB 11.5m x 7.5m x
8.5m anechoic chamber, large hangar shed, open area test site and vibration
room for testing transmitters and receivers.

\subsection*{\it Cavendish Observatory, Cambridge}

1PhD student x 3 years  for exploring imaging/calibration using standard software
0.5 Postdoc x 2 years for novel power spectral approaches
0.3 Engineers x 3yrs for field work and system testing
0.2 FTE Carilli

\subsection*{\it University of California, Berkeley}

0.25 FTE DeBoer for 4 years
0.25 FTE MacMahon for 4 years
0.25 FTE Dexter for 4 years
1.5 months Parsons summer salary for 5 years
laboratory correlator system for development and testing of correlators
network analyser for characterizing dish reflections
henyey compute cluster for local data processing
HFSS electromagnetic modelling software
Matlab
Blue Jean video conferencing software
PAPER's delay spectrum power spectrum pipeline software
machine tools
HERA test antenna

% XXX update

The UC Berkeley Radio Astronomy Laboratory (RAL) is located adjacent to the UC
Berkeley Astronomy Department. It provides laboratory space and access to
digital and radio-frequency test equipment necessary for the detailed
characterization and performance testing of components of the proposed correlator
development work. Such items include power supplies, signal generators, network
analyzers, oscilloscopes, noise generators, filters, attenuators, amplifiers,
and other miscellaneous electronic equipment. RAL also hosts the programming
environment targeting the Field-Programmable Gate Array processors on which the
proposed correlator work is based. This facility is available to members of the
UC Berkeley Astronomy Department.

The Berkeley Wireless Research Center is located adjacent to the UC Berkeley
campus. It provides access to additional high-end digital test equipment and
software for digital signal processing.  Available to members of the
Collaboration for Astronomy Signal Processing and Electronics Research.

\subsection*{\it University of Pennsylvania}

The central computing cluster and data archive for the Precision Array for Probing the Epoch of Reionization (PAPER) is maintained at the University of Pennsylvania (Penn).  This currently consists of 22 nodes connected by Gb ethernet: 16 Dell PowerEdge 1950s (dual 4-core Intel Xeon L5420 \@ 2.50GHz), and 6 Dell PowerEdge R410 (32 GB RAM; dual 6-core Intel Xeon E5649 \@ 2.53GHz), for a total of 200 cores and 448 GB RAM.  Fast working data storage is provided by a network storage system (NSS) based around two Dell MD1200s, with a total RAID storage of 140 TB.  The storage has a dedicated head node and internal 10Gbe, with 1 Gbe connections to each compute node. Backup of the NSS is done on to several Silicon Mechanics Storform D59J.v2 machines, each of which is about 55 TB effective in RAID6.  We currently have three backing up the 140 TB above. The system was purchased through a combination of the PI's startup funds, internal funding from Penn's University Research Foundation, a major contribution from the Mt.~Cuba Astronomical Foundation, and contributions from Professors M. Sako and G. Bernstein, who share the cluster for processing of DES and SDSS data.  Extra space is available to house the additional data analysis and storage hardware described in this proposal to support the HERA array.


\subsection{\it University of Washington}

The University of Washington is providing software developed as a part of the MWA, including Monitor \& Control software, Fast Holographic Deconvolution software, and direct power spectrum analysis software, in addition to a RF testing laboratory (vector network analyzers, spectrum analyzers, mixed mode oscilloscopes, precision Tsys measurements, etc.) and access to an excellent machine shop.

Morales led the development of the MWA Monitor \& Control software, and this will provide a strong code base for developing the associated system for HERA. Morales' group has also been at the forefront of widefield imaging and foreground subtraction, and has developed the Fast Holographic Deconvolution software that is being used to reduce PAPER imaging data and all MWA Epoch of Reionization data. This code is crucial to reaching the scientific goals of HERA, including the survey and instrumental commissioning with HERA 37 and obtaining the foreground isolation necessary for the ultimate PS and imaging of the EoR with the full HERA instrument.


\end{document}
