\documentclass[11pt]{article}
\usepackage{fullpage}

%Facilities, Equipment, and Other Resources: In order for NSF, and its
%reviewers, to assess the scope of a proposed project, all organizational
%resources necessary for, and available to a project, must be described in this
%section of the proposal (see GPG Chapter II.C.2.i for further information).
%Proposers should describe only those resources that are directly applicable.
%Proposers should include an aggregated description of the internal and external
%resources (both physical and personnel) that the organization and its
%collaborators will provide to the project, should it be funded. Such
%information must be provided in this section, in lieu of other parts of the
%proposal (e.g., budget justification, project description). The description
%should be narrative in nature and must not include any quantifiable financial
%information.

\begin{document}
\pagestyle{empty}

\section*{Facilities, Equipment, Other}

The work described in this proposal will take place at XXX locations: 

\subsection*{\it Arizona State University}

Octocopter system
electronics development and fabrication shop
total power measurement and calibration system developed for EDGES

\subsection*{\it Massachussetts Institute of Technology}

use of MWA data archive?
Omniscope FFT-correlator design and hardware,
Omniscope antenna array for prototyping hexagonal FFT correlator
MWA imaging and calibration pipeline

\subsection*{\it NRAO} 

Bradley support
Carilli support
Green Bank site
Infrastructure and systems associated with the PAPER-32 array deployed in Green Bank.
CST software for simulating feed responses

% XXX update

Our 500 sq. ft. field station is located on the NRAO Green Bank, WV observatory
site. It houses a work area and plenty of space to deploy instrumentation.
Positioned within the National Radio Quiet Zone, this site is an ideal venue
for very sensitive radio frequency measurements due to its remoteness from
large urban areas and its exceptional laboratory infrastructure. A relatively
large amount of outdoor space is available adjacent to the station for the
deployment of antennas under test. The satellite downlink measurement system is
here, and the PAPER array is located nearby. Electromagnetic enclosures are
used to house all instrumentation to prevent self-interference as well as
interference with Observatory telescopes, in strict adherence to the radio
emissions control policy of the Observatory’s Interference Protection Group
(IPG).

There warehouse facilities located in Green Bank are ideal for staging our
shipments to South Africa. Wooden crates are built in the carpenter shop and
moved to the warehouse for packing and short term storage. Heavy lift
equipment and the expertise to use them safely are also available to transport
these crates to the truck mounted shipping containers destined for the cargo
ship.

Our 800 sq.ft. research laboratory is located at the NRAO’s Technology Center.
It is equipped with two complete micro-assembly workstations for component
fabrication and rework as well as a suite of basic instruments necessary for
evaluation and repair. In addition, the laboratory contains several signal
generators, power meters, spectrum analyzers, an impedance bridge, network
analyzers, and noise figure meters for radio frequency applications. A small
anechoic chamber is available for antenna impedance measurements, and an
computer-controlled environmental chamber is ready for studies of thermal and
humidity induced effects on component and subsystem designs. We routinely make
use of the NRAO machine shop facilities.

We also have access to the latest computer-aided design software. Autocad
Inventor is used for three-dimensional mechanical drawings. Agilent’s Advanced
Design System (ADS) is available for circuit and subsystem modeling while CST
Microwave Studio is used for electromagnetic simulations of antenna and other
RF structures. In addition, CF Design is available for modeling thermal
conduction and radiation pathways.

\subsection*{\it PAPER}

PAPER 128 dual-polarization correlator, including 16 ADCx16 boards, 
8 ROACH-II processing boards, 8 servers each hosting 2 GTX690 dual-GPU cards,
and two data acquisition servers.
Data compression and storage system, including 5 computing servers and 
a XXX TB raid storage system.
128 PAPER elements, with analog signal chains, including nodes, receivers.
EMC enclosure housing the correlator and electronics.
Power infrastructure (transformers, UPS)
HVAC system.

\subsection*{\it South Africa}

Site clearing
Use of KAPB
Power + internet service
power + internet infrastructure
SKASA liason
accommodations
occasional operations maintenence
access roads to site
Bernardi support

% XXX update

PAPER’s been given the use of a 300m diameter cleared, level, circular area
located within the protection of South Africa’s Astronomy Geographic Advantage
Act. A 120dB RFI shielded 12m container has been placed in the center to house
electronics. The entire system is powered by 20kVA diesel generator though
installation of a grid connection is under investigation. The container houses
work space and two 19" 42U racks for PAPER systems. It is cooled by a 5 kW
capacity chiller plant. This should provide sufficient capacity for up to a 128
antennas.

In addition, SKA-SA assists with the unloading, assembly and storage of
shipping crates. Two on-site mobile truck cranes are available along with use
of a 30m x 18m dish construction shed, shared with the KAT-7 construction.
Accommodation is provided for on-site PAPER visitors and off-road pickup trucks
are available for use to, from and while on site.

A digital electronic laboratory is available in the Cape Town SKA-SA offices,
complete with hot-air rework stations and high frequency test equipment
(bench power supplies, signal generators, network analyzers, spectrum
analyzers, oscilloscopes up to 100GHz etc). There is an off-site RFI
measurement facility near Cape Town (Houwteq) able to perform measurements to,
amongst others, MIL-STD-461E (18GHz) with a greater than 100dB 11.5m x 7.5m x
8.5m anechoic chamber, large hangar shed, open area test site and vibration
room for testing transmitters and receivers.

\subsection*{\it Cavendish Observatory, Cambridge}

\$0.5M (can't be enumerated... what can this buy?)
grad + (maybe) postdoc
0.2 FTE Carilli

\subsection*{\it University of California, Berkeley}

0.25 FTE DeBoer for 4 years
0.25 FTE MacMahon for 4 years
0.25 FTE Dexter for 4 years
1.5 months Parsons summer salary for 5 years
laboratory correlator system for development and testing of correlators
network analyser for characterizing dish reflections
henyey compute cluster for local data processing
HFSS electromagnetic modelling software
Matlab
Blue Jean video conferencing software
PAPER's delay spectrum power spectrum pipeline software
machine tools
HERA test antenna

% XXX update

The UC Berkeley Radio Astronomy Laboratory (RAL) is located adjacent to the UC
Berkeley Astronomy Department. It provides laboratory space and access to
digital and radio-frequency test equipment necessary for the detailed
characterization and performance testing of components of the proposed correlator
development work. Such items include power supplies, signal generators, network
analyzers, oscilloscopes, noise generators, filters, attenuators, amplifiers,
and other miscellaneous electronic equipment. RAL also hosts the programming
environment targeting the Field-Programmable Gate Array processors on which the
proposed correlator work is based. This facility is available to members of the
UC Berkeley Astronomy Department.

The Berkeley Wireless Research Center is located adjacent to the UC Berkeley
campus. It provides access to additional high-end digital test equipment and
software for digital signal processing.  Available to members of the
Collaboration for Astronomy Signal Processing and Electronics Research.

\subsection*{\it University of Pennsylvania}

Power and operations of computing cluster and data storage.
use of Current PAPER computing cluster and storage

J. Aguirre’s laboratory at the U. Pennsylvania contains signal processing
equipment, hard drives, and data backup systems necessary for the analysis of
PGB-32 and PSA-32 array data. Extra space is available to house the additional
data analysis hardware described in this proposal to support the PSA-128 array.


\subsection{\it University of Washington}

MWA monitor and control Data base system
Mechanical shop
FHD holographic deconvolution software

M. Morales' laboratory at U. Washington provides access to precision radio
development laboratory equipment, including
a vector network analyzer, spectrum analyzer, mixed-mode oscilloscope, and
precision power and clock equipment.  This laboratory also provides computing
resources for software development, testing, and basic data analysis.

\end{document}
