\documentclass[preprint]{aastex} 

\usepackage[top=1in, bottom=1in, left=1in, right=1in]{geometry}
\setlength{\parskip}{0pt}
\setlength{\parsep}{0pt}
\setlength{\headsep}{0pt}
\setlength{\topskip}{0pt}
\setlength{\topmargin}{0pt}
\setlength{\topsep}{0pt}
\setlength{\partopsep}{0pt}
\setlength{\footnotesep}{8pt}
\pagestyle{empty}

\begin{document}
\title{Supplementary Document:  Partner Institutions and Roles}

\noindent
Listed below are the partner institutions that are to be funded
via subawards, along with the role of each in the project.
\vspace{0.25in}

\section*{University of California, Berkeley}

issue contracts for fab, shipping, and deployment of antennas, analog
electronics, nodes, correlator, and infrastructure.  Issue subcontracts to
other institutions.  Develop node, correlator, and dish design, and code for
compressing data on correlator hardware.  Develop \& apply software for power
spectrum data analysis based on AIPY/delay spectrum/covariance diagonalization
techniques.  hardware commissioning delay-based power spectrum software
pipeline cross-talk phase switch in node

\section*{University of Pennsylvania}

deploy field data management system (disks, servers, etc.) that interfaces to
correlator hardware for data compression and storage.  Upgrade and support
computing cluster \& storage for data analysis.  Perform polarization-oriented
software development and data analysis, characterizing polarized sky.
computing and data storage/transport systems quality assurance systems
polarized sky and leakage evaluation

\section*{Massachusetts Institute of Technology}

support site manager (penciled in as Goeke?), develop hex FFT correlator
prototype in GB, which will be adopted into SA deployment as available.
Perform data calibration, sanity checks, and imaging.  Develop and apply
optimal estimator techniques for power spectrum analysis.  Characterization of
low-frequency sky \& prospects for dark-ages science.  optimal estimators data
product distribution operational data analysis manufacturability optimization \&
project engineering transient science low-frequency dark ages science
cross-correlation with other data sets

\section*{University of Washington}

develop and apply FHD \& related techniques for imaging and power spectrum
analysis, focusing on direction-dependent gain/leakage issues.  Characterize
primary beam based on celestial sources.  monitor/control client calibration
and model subtraction software subtraction-based power spectrum pipeline

\section*{National Radio Astronomy Observatory}

transition PAPER analog electronics to 50Ohm, develop new feed optimized for
dish illumination and minimization of polarization leakage, prototyped on 2
HERA dishes deployed in GB.  Perform foreground \& EoR science based on direct
imaging with CASA/AIPS software.  Develop CASA software with enhancements for
imaging with wide-field, wide-band, etc. data.  feed development foreground
science support for CASA software

\section*{Arizona State University}

Develop and apply In-situ beam calibration systems (octocopter?) and develop
some absolutely-calibrated balun/receivers.   Application of higher-order
statistics to 21cm reionization science.  science commissioning total-power
balun development octocopter beam calibration data calibration


\section*{University of California, Los Angeles}

Furlanetto will contribute to the HERA project during the final stage of data
analysis, building a tool to bridge the gap between the data products and the
physics questions of galaxy formation and evolution. Specifically, Furlanetto
will provide a grid of simulations of the sources of the spin-flip background,
identifying and varying the key parameters. The experimental data will then be
compared to this grid in order to extract quantitative constraints on these
parameters. This work will be done by Furlanetto and a graduate student (to be
named), for whom this project will constitute their PhD thesis.

\end{document}
