\documentclass[preprint]{aastex} 

\usepackage[top=1in, bottom=1in, left=1in, right=1in]{geometry}
\usepackage{amsmath}
\usepackage{graphicx}
\usepackage{mdwlist}
\usepackage{natbib}
\usepackage{natbibspacing}
\setlength{\bibspacing}{0pt}
\setlength{\parskip}{0pt}
\setlength{\parsep}{0pt}
\setlength{\headsep}{0pt}  
\setlength{\topskip}{0pt}
\setlength{\topmargin}{0pt}
\setlength{\topsep}{0pt}
\setlength{\partopsep}{0pt}
\setlength{\footnotesep}{8pt}
\pagestyle{empty}
\citestyle{aa}

\def\kperp{k_{\bot}}
\def\kpar{k_{\|}}
\def\k{{\bf k}}
\def\sky{{\theta}}
\def\HI{{H{\small I }}}

%\usepackage{subfig}
%\usepackage[countmax]{subfloat}

%Project Description (8-pages maximum), including the following:
%- A statement of which of the four categories of MSIP is most appropriate
%for this proposal as the first sentence (see section II. Program Description).
%- A scientific justification. For Open Access Capabilities, explain the
%uniqueness and lack of general availability of the capability.
%- A description of the broader impacts, including student training.
%- A description of benefits to the community (observing time, data products, etc.)
%- An outline of the project management plan (where appropriate).
%Note: Results from Prior NSF Support should not be included. Links to URLs may
%not be used.
\begin{document}
%\title{Hydrogen Epoch of Reionization Array}
\title{HERA: Illuminating Our Early Universe\\
{\it In the Mid-Scale Science Projects category of the Mid-Scale Innovations Program}} 

% A statement of which of the four categories of MSIP is most appropriate
%for this proposal as the first sentence (see section II. Program Description).

%This proposal targets the Mid-Scale Science Projects category of the Mid-Scale Innovations Program solicitation. The Hydrogen Epoch of Reionization Arrays (HERA) is a program for using the unique capabilities of the 21cm hyperfine line to trace neutral hydrogen through the cosmic dawn of our Universe.  The HERA roadmap that was submitted to the {\it New Worlds, New Horizons of Astronomy and Astrophysics} 2010 decadal survey, (hereafter NWNH) was given ``top priority in this [Radio, Millimeter, and Sub-millimeter] category of recommended new facilities for mid-scale funding." The HERA roadmap proceeded in three stages: HERA-IB called for \$25M to complete the PAPER and MWA experiments; HERA-II budgeted \$62M for an array with 0.1 km$^2$ of collecting area capable of characterizing the power spectrum of cosmic reionization in detail; HERA-III targeted 1 km$^2$ of collecting area to image reionization structures in detail.

{ \setlength{\parindent}{0cm}
The major stages in the history of our Universe are written in the phases of hydrogen. The Hydrogen Epoch of Reionization Arrays (HERA) roadmap is a staged program that uses the unique properties of the 21~cm line from neutral hydrogren to probe the evolution of the intergalactic medium (IGM) during `cosmic reionization' and the preceding `dark ages'. Cosmic reionization corresponds to the epoch when the first stars and black holes reionize the neutral IGM that pervaded the Universe following cosmic recombination, roughly 0.5~Gyr to 1~Gyr after the Big Bang. This epoch represents the last frontier in studies of cosmic structure formation. Direct observation of the evolution of large scale structure via the HI 21~cm line will have a profound impact on our understanding of the birth of the first galaxies and black holes, their influence on the IGM, and cosmology.  HERA was given the ``top priority in the Radio, Millimeter, and Sub-millimeter category of recommended new facilities for mid-scale funding" as part of the {\it New Worlds, New Horizons of Astronomy and Astrophysics} decadal survey, (\citealt{astro2010}; hereafter NWNH).  
}

The HERA roadmap envisioned a series of radio interferometers constructed throughout the decade, starting with the PAPER and MWA instruments (Donald C. Backer Precision Array for Probing the Epoch of Reionization; Murchison Widefield Array) aimed at characterizing foregrounds and a first effort to detect the Epoch of Reionization (EoR) power spectrum, a second-generation instrument to measure the EoR power spectrum in detail and reveal how early structure in the universe formed, and a third-generation instrument late in the decade to image the EoR. 

Using the advances spearheaded by the MWA and PAPER experiments, we are proposing to build a 568-element HERA instrument, observing in the 50--250 MHz band, that not only fulfills the goal of detailed power spectrum characterization as a second-generation instrument, but also (in its late stages) is capable of imaging the EoR, a task previously considered possible only for third-generation instruments.
%HERA will observe in the 50--250 MHz band to extract the exciting science offered by 21~cm observations of the first stars and galaxies.
In \S \ref{SJsec} we describe the science HERA enables. Section \ref{LessonsSec} then reviews the breakthroughs in our understanding of the EoR foregrounds from PAPER and MWA, and the techical heritage that allows HERA to perform the science envisioned in the decadal survey at significantly reduced cost. The full HERA instrument and timeline is described in \S \ref{PDsec}, followed by the impacts on the US scientific and national community in \S \ref{BIsec}.

\vspace{-0.25in}
\section{Scientific Justification}
\label{SJsec}

The period beginning with the birth of the first stars, and culminating with
the full ionization of the intergalactic medium (IGM) $\sim$500 Myrs later, is
the last major unexplored phase in the evolution of luminous structures in the
Universe.  Whether during the ``Dark Ages" (z$\ge$15) or cosmic reionization
(z$\sim$15--6), a wealth of astrophysical and cosmological phenomena are at
play.  The precise properties of the IGM depend on the nature and distribution
of the first luminous sources (eg. typical masses, UV escape fractions, biased
structure formation), the efficiency and abundance of heating sources (eg.
X-ray binaries, shocks, or even dark matter annihilations), the formation of
the first supermassive black holes, and the relative velocity of baryonic
matter and dark-matter halos, among other effects.  Exploration of the Dark
Ages and the Epoch of Reionization (EoR), and in particular, the evolution of
the IGM during these epochs, has been called-out as one of the top three
``priority science objectives chosen by the [NWNH] survey committee for the
decade 2012-2021".

Thus far, a number of indirect probes have been used to understand cosmic
reionization.  These include observations of Gunn-Peterson attenuation by the
IGM toward the most distant quasars \citep{fan_et_al2006,bouwens_et_al2010},
kinetic Sunyaev-Zel'dovich features in the CMB \citep{zahn_et_al2012}, CMB
anisotropy and polarization \citep{page_et_al2007,planck_et_al2013}, and the
demographics of Ly$\alpha$ emitting galaxies
\citep{http://adsabs.harvard.edu/abs/2013arXiv1308.5985T}.  Unfortunately,
these ground-breaking results are limited in diagnostic capabilities: the
Gunn-Peterson effect and related phenomena, saturate at low neutral fractions,
and the CMB provides only an integral measure of the Thompson optical depth
back to recombination.  Moreover, many of these indirect observations are in
tension with one another, underscoring both the difficulty in interpreting
their results and the fact that reionization was a complex process (see Fig \ref{fig:x_i_Xray}).
A direct probe of emission from the neutral IGM is clearly required to fully
unravel the complex processes involved in cosmic reionization. 

The 21~cm hyperfine transition has been recognized as potentially the most
powerful probe of the evolution of the IGM during cosmic reionization, and into
the preceding `Dark Ages' \citep{morales_wyithe2010,furlanetto_et_al2006}.
Detecion of this signal will have an impact
comparable to the discovery of the CMB, and study of the three-dimensional
evolution of large scale structure in the IGM via the HI 21cm line has the
potential to become `the richest of all cosmological data sets'
\citep{barkana_loeb2005a,loeb_zaldarriaga2004}).  As emphasized in
NWNH \citep{astro2010}: ``The panel concluded that to explore the discovery
area of the epoch of reionization, it is most important to develop new
capabilities to observe redshifted 21~cm HI emission, building on the legacy of
current projects and increasing sensitivity and spatial resolution to
characterize the topology of the gas at reionization."  Although early 21~cm
EoR experiments with limited sensitivity are targeting statistical detections
of the power spectrum of reionization, the 21~cm signal versus redshift can
eventually be reconstructed into three-dimensional maps of the evolution of
cosmic structure that would greatly improve our understanding of the
cosmological and astrophysical evolution of the universe
\citep{furlanetto_et_al2006,mao_et_al2008,morales_wyithe2010}. As a high
sensitivity instrument with broad frequency coverage, HERA would be capable of
painting a consistent and uninterrupted picture of not just the EoR, but also
into the preceding Dark Ages.  

%%%

%The period beginning with the birth of the first luminous objects in the
%universe, and culminating with the ionization of the intergalactic medium (IGM)
%$\sim$500 Myrs later, is one of the last unexplored phases of cosmic evolution.
%The NWNH decadal survey highlighted the exploration of the epoch of reionization as %one of the three
%``priority science objectives chosen by the [NWNH] survey committee for the
%decade 2012-2021". Current observations of Gunn-Peterson absorption
%by the IGM toward the most distant quasars \citep{fan_et_al2006,bouwens_et_al2010}
%kinetic
%Sunyaev-Zel'dovich features in the CMB \citep{zahn_et_al2012}, and CMB
%anisotropy and polarization \citep{page_et_al2007,planck_et_al2013} are
%moderately in tension with one another, suggesting
%that reionization was a complex process.  Unfortunately, these ground-breaking %results are limited in
%diagnostic capabilities: the Gunn-Peterson effect saturates for even low
%neutral fractions, and the CMB provides only an integral measure of the
%Thompson optical depth back to recombination.
%
%Redshifted emission from the 21~cm hyperfine transition of neutral hydrogen has
%gained considerable attention as a unique tracer of the
%primordial IGM.  Directly observing the neutral IGM via this signal
%would be an achievement comparable with the discovery of the CMB.  As
%emphasized in NWNH \citep{astro2010}: ``The panel concluded that to explore the
%discovery area of the epoch of reionization, it is most important to develop
%new capabilities to observe redshifted 21-cm HI emission, building on the
%legacy of current projects and increasing sensitivity and spatial resolution to
%characterize the topology of the gas at reionization."
%Although early 21~cm EoR experiments with limited sensitivity are targeting
%statistical detections of the power spectrum of reionization, the 21~cm signal
%versus redshift and angle can eventually be reconstructed into
%three-dimensional maps of the evolution of cosmic structure that would greatly %improve
%our understanding of the cosmological
%and astrophysical evolution of the universe
%\citep{furlanetto_et_al2006,mao_et_al2008,morales_wyithe2010}.
%
%During the ``Dark Ages" (z$\sim$150--15) and cosmic reionization (z$\sim$15--6),
%21~cm emission is sensitive to a wealth of astrophysical and cosmological phenomena.

The evolution of the HI 21cm signal from the neutral IGM depends on miriad
physical processes, including: general large scale structure evolution, IGM
ionization by the first galaxies and black holes, and the complex interplay
between the gas kinetic and excitation temperatures, and the temperature of the
CMB \citep{furlanetto_et_al2006}. The gas kinetic temperature can be affected
by shocks or pervasive X-rays from the first luminous sources, and the HI
excitation temperature can be dictated by collisions, CMB photons, or resonant
scattering of ambient Ly$\alpha$ photons, depending on epoch
\citep{pritchard_loeb2012}.  The relative importance of these competing effects
is sensitive to, among other things, the expansion of the universe, the
ignition of the first stars and galaxies, the formation of the first massive
black holes \citep{mesinger_et_al2013}, and the relative velocity of baryonic
matter and dark-matter halos \citep{mcquinn_oleary2012}.  

Over the past decade, cosmologists have devoted considerable effort to modeling the complex astrophysics of reionization. The resultant simulations have grown in
sophistication \citep{santos_et_al2010,mesinger_et_al2011,wyithe_loeb2004}, but still 
have difficulty bridging the enormous scale difference between the self-shielding
regions that are the primary sinks of ionizing photons and volumes required for statistically
representative samples of cosmic structures.
Most importantly, basic constraints on cosmic reionization remain rudimentary at best. When did it occur, and over what timescale?  What objects dominated the radiation field? How were the objects distributed? Did the first generation of stars \& galaxies enhance or suppress the formation of subsequent stars \& galaxies in the original halo and smaller nearby halos? Without these basic contraints, 
further progress on theoretical modeling of first galaxy formation and cosmic reionization remains problematic. 

The HERA program as proposed provides a powerful new tool which will generate critical constraints on cosmic reionization, and potentially on large scale structure evolution during the dark ages. HERA is designed such that science return will be realized through-out the build-out of the project. The main science goals over time for the project are:

% NOTE: IN THE FOLLOWING MANY OF THE MAIN ISSUES ARE THERE, BUT PERHAPS AT THE 
% WRONG TIME WRT BUILD-OUT? 

\begin{itemize}

\item Years 1 \& 2: Perform deep continuum survey using HERA-37 in the target reionization fields for bright source characterization. 

\item Year 3: Hera~127 will have sufficient sensitivity to determine of the evolution of the reionization power spectrum. The results will determine the primary epoch of cosmic reionization (the `50\% neutral fraction zone'), and the length of time during which reionization `bubbles' dominated the HI power spectrum. These data will also place strong constraints on warming of the neutral IGM by X-rays, and hence the rate and density of black hole formation in the
early universe (\citealt{pritchard_loeb2010}; see Figure \ref{fig:x_i_Xray}, left panel). Note that study of the neutral IGM during the key redshift range from $z$ = 8 to 10 is difficult using standard techniques based on eg.\ Ly$\alpha$ spectra or galaxy populations due to saturation of the Ly$\alpha$ attenuation. HERA is uniquely suited to probe this redshift gap (see Figure \ref{fig:x_i_Xray}, right panel). 

\item Year 4: HERA~331 will make high sensitivity power spectral measurements over a significant range in wavenumber. The detailed shape of the power spectrum will be measured, dictating the nature and distribution of the first galaxies that dominate cosmic  reionization. Is early galaxy formation highly biased, or more uniformly distributed (inside-out vs.\ outside-in reionization)? Is reionization dominated by low or high mass galaxies? What is the escape fraction of UV photons in early galaxies? How does feedback from early star formation affect low-mass galaxies? [former might be overstating things?] The change in the slope of the power spectrum of 21~cm emission through reionization (Figure \ref{fig:eor_pspec}) determines the size of ionization bubbles at various stages of reionization, which constrains the relative contributions to the ionizing background of halos as a function of mass, and helps us understand how efficiently early structures cooled and formed stars. This array will also provide the first images of the largest scale structures during reionization, and determine how velocity streaming between baryonic  matter and the dark matter halos affected early structure formation and the onset of Ly$\alpha$ emission \citep{visbal_et_al2012}.
% XXX does this check for HERA-331?

\item Year 5: HERA 568 will be a powerful instrument to characterized the power-spectrum through the reionization into the dark ages, as well as for direct imaging of the IGM during reionization. Imaging will be particularly important for comparison with large scale galaxy surveys that extended into cosmic reionization (see section ?? below). HERA-568 will have the sensitivity to push toward higher redshifts (lower frequencies), into the cosmic dark ages. More on Xray heating and dark matter annihiliation dark ages here...
% XXX 

\end{itemize}


%The new window into high-redshift 21~cm observations provided by HERA
%will begin to explore the rate and density of massive black holes formed in the
%early universe \citep{pritchard_loeb2010} via their X-ray emission (see Figure \ref{fig:Xray}), 
%how velocity streaming between baryonic 
%matter and the dark matter halos affected early structure formation and the onset
%of Ly$\alpha$ emission \citep{visbal_et_al2012}, and will lay the groundwork for future
%efforts to explore how 
%cosmological models can be improved via measurements of redshift-space distortions,
%artificial anisotropies introduced via the Alcock-Paczy\'inski effect, and
%gravitational lensing signals\citep{furlanetto_et_al2006}.
%As illustrated in Figure \ref{fig:x_i}, observing the 21~cm line 
%through this epoch with HERA-568 
%promises to determine the ionization history of our universe much more precisely,
%and at higher neutral fractions, than is possible with other existing techniques.  These measurements can
%be used to move beyond characterizing the timing and duration of reionization to
%explore which galaxies dominate the integrated UV luminosity density, what the escape fraction
%of UV photons is in early galaxies, and how feedback from early star formation affects low-mass galaxies and the integrated global %ionization profile versus
%redshift.  Moreover, measurements of the slope of the power spectrum of 21~cm emission through
%reionization (Figure \ref{fig:FourPanContour}) determine the size of ionization bubbles for at
%various stages of reionization, which
%constrains the relative contributions to the ionizing background of halos as a function of mass,
%and helps us understand how efficiently early structures cooled and formed stars.

Cosmic reionization sits at the frontier of modern astrophysics and cosmology. With the exception of observations of a few
of the brightest galaxies and AGN, we have no detailed measurements of this phase in the history of our universe.  
While it is reasonable to forecast how observations with HERA will refine our understanding based on current
models, these models are necessarily incomplete.  There is an exciting possibility that 
the measurements produced by HERA could depart substantially from what we expect.  The new
windows that HERA opens into our early universe have the capability to transform our scientific
understanding of the complex and fascinating intersection of cosmology and astrophysics, and to
help develop the widely recognized scientific potential of 21~cm cosmology.

%\begin{figure}[!ht]\centering
%\includegraphics[width=6in]{plots/21cm_cosmo.jpg}
%\caption{\small
%The 21cm hyperfine line of neutral hydrogen represents the next frontier in
%precision cosmology. Following recombination (left edge), the spin temperature
%of 21cm emission is sensitive to the density and temperature of the
%intergalactic medium (IGM) through the Dark Ages, evolves with the heating and
%ionization of the IGM during the Epoch of Reionization (EoR), and
%traces the distribution of galaxies as the universe expands,
%leading up the the present day (right edge).  Color indicates
%redshift, which stretches the 21cm line to frequencies ranging from 50 MHz
%(red) to 1.4 GHz (violet).
%}\label{fig:21cm_cosmo}
%\end{figure}

\begin{figure}[!ht]\centering
\includegraphics[width=3in]{plots/constraints.png}
~ % gap
\includegraphics[width=3in]{plots/Xray.pdf}
\caption{\small
%[XXX: Replace with plot that contains current constraints on $x_i$ from other probes.]
Left: With the help of theoretical models, a measurement of the $21\,\textrm{cm}$ power spectrum
can be turned into a timeline of our Universe's ionization state, $x_i$.  At high redshifts, 
$21\,\textrm{cm}$ arrays will be the only probe of $x_i$, while at low redshifts the weaker
foregrounds can produce constraints that improve upon other probes of reionization.
%[XXX: Update this plot when the Aaron/Josh comparison is done.] 
Right: HERA's ability to
observe at relatively high sensitivity even at low frequencies opens a window to
higher redshift pre-reionization physics.  Shown here as a function of redshift are power spectrum amplitudes
for a variety of theoretical models that differ in their IGM heating prescription (involving various X-ray heating
and dark matter annihilation scenarios, for example).  Sensitivities for various instruments are also plotted.
With continuous spectral coverage, it is likely that HERA will be sensitive to a high redshift peak corresponding
to X-ray heating, in addition to the lower redshift peak that is driven by ionization fraction.
}\label{fig:x_i_Xray}
\end{figure}

\begin{figure}[!ht]\centering
\includegraphics[height=2.25in]{plots/eor_pspec.png}
~ % gap
\includegraphics[height=2.25in]{plots/hera_snr_contour.png}
\caption{\small
Left: Predicted power-spectrum sensitivities for three stages of
HERA (solid) relative to a fiducial 50\% ionization model 
(dashed; \citealt{lidz_et_al2008}).  Sensitivities reflect
180-day drift-scan observations of a 6-hour patch of sky, excluding,
as a function of baseline length, modes that are expected to be
dominated by systematics, as described in \S\ref{LessonsSec}.  In addition
to a staged array size, sensitivity curves follow
a staged improvement in analysis software that expands the range
of modes that are not corrupted by systematics.  
Right: 
The rise and fall of the 21~cm power spectrum predicted in
\citep{lidz_et_al2008} as a function of an ionization fraction, $x_i$,
that evolves with redshift.  Contours indicate the predicted significance
of detection with HERA-568, which constrain ionization fraction, as shown in
Figure \ref{x_i_Xray}.  The dashed line dashed line indicates the location
of the model used in the left panel.
}\label{fig:eor_pspec}
\end{figure}

%\begin{figure}[!ht]\centering
%\includegraphics[width=4in]{plots/FourPanContour.png}
%\caption{\small
%[XXX: Replace with a plot that has the proper axes and color %bar.  Make some statement
%about detection at the earliest stages as well as the potential to do more.
%}\label{fig:FourPanContour}
%\end{figure}

\vspace{-0.25in}
\section{Lessons learned from PAPER and MWA}
\label{LessonsSec}

The challenge of 21~cm cosmology is isolating the faint EoR signal from astrophysical foregrounds that are 4-5 orders-of-magnitude brighter, as seen in the lefthand panel of Figure \ref{fig:twoFGViews}. The major breakthrough in 21~cm cosmology---what enables us to propose HERA now---is the discovery of the EoR Window.

21~cm cosmology observations and foreground isolation are best understood in the three dimensional wavenumber space $\k$.
Because the \HI emission is a narrow spectral line, the observed frequency of the emission can be mapped to redshift or line-of-sight distance to provide an observed volume $\{x,y,z\}$ in cMpc. This observed volume is Fourier transformed into a three dimensional wavenumber cube $\{k_{x}, k_{y}, k_{z}\}$.\footnote{Interferometric measurements are of the angular Fourier modes in many frequency channels (visibilities), so in the absence of widefield effects only a Fourier transform in the frequency direction and a coordinate mapping is needed to obtain the 3D $\{k_{x}, k_{y}, k_{z}\}$ measurements \citep{morales_hewitt2004}.}
Spatial isotropy allows measurements within this 3D wavenumber space to be squared and averaged in shells to produce the spherical power spectrum presented in Figure \ref{fig:eor_pspec}, and for graphical simplicity the angular wavenumbers are typically averaged ($\{k_{x},k_{y}\}\rightarrow\kperp$) to produce line-of-sight wavenumber $\kpar$ vs.\ angular $\kperp$ figures.

The astrophysical foreground emission is spectrally very smooth (synchrotron \& Bremsstrahlung emission) or at known editable frequencies (radio recombination lines). The advance in 21~cm cosmology has been understanding how this foreground emission interacts with the instrument to produce the EoR Window. Through a concerted theoretical and observational campaign \citep{morales_et_al2012,parsons_et_al2012b,vedantham_2012,hazelton_et_al2013,pober_et_al2013,parsons_et_al2013,dillon_et_al2013b} we now understand that the foreground contamination is confined to a `wedge' in $\kpar$ vs.\ $\kperp$, as demonstrated by the PAPER observations in the righthand panel of of Figure \ref{fig:twoFGViews}. This wedge is the result of the smooth spectrum foregrounds (low $\kpar$) interacting with the inherent chromaticity of an interferometer. This leaves the region above isolated from the foreground emission---a window through which we can observe the EoR.

All observations have now confirmed the presence of the EoR Window \citep{pober_et_al2013,dillon_et_al2013b}, and this advance has lead to the first meaningful costraints on 21~cm emission from the EoR in \citet{parsons_et_al2013}. The MWA and PAPER teams have been at the forefront of developing the EoR Window, writing all of the papers in the literature and developing the delay-spectrum and imaging power spectrum analyses to exploit this insight. 




\begin{figure}[t]
\centering
%\includegraphics[width=6.5in]{plots/MWApretty.png}
\includegraphics[width=3.6 in]{plots/MWApretty.png}
~ %gutter btwn graphics
\includegraphics[width=2.4 in]{plots/wedge_tall.png}
\caption{\small
The left panel shows foregrounds as imaged with the MWA using the Fast Holographic Deconvolution software package developed by the University of Washington \citep{sullivan_et_al2012}. This image uses 2 minutes of MWA data towards galactic anti-center, spans $\sim$$30^{\circ}$ with the Vela and Puppis SNRs in the lower righthand corner, and has less than $0.5\%$ polarization leakage across the field. The right hand panel shows the EoR window as observed by PAPER \citep{pober_et_al2013}. While astrophysical foregrounds are very bright as seen in the `wedge' to the bottom right, their contribution falls by orders-of-magnitude to below the thermal noise in the EoR window as predicted by \citep{morales_et_al2012,parsons_et_al2012b,vedantham_2012,hazelton_et_al2013}.
}\label{fig:twoFGViews}
\end{figure}

%\begin{figure}\centering
%\includegraphics{plots/Pober_wedge.pdf}
%\caption{\small
%The EoR window as measured by PAPER.  Blue represents a region containing EoR signal but no foreground contamination.
%}\label{fig:EoRwindow}
%\end{figure}

%\begin{figure}[!ht]\centering
%\includegraphics[width=1.0\textwidth]{plots/twoFgViews.png}
%\caption{\small
%[XXX: Redo in a more attractive way.]
%Left: Foregrounds in the image domain, as observed by the MWA using the Fast Holographic Deconvolution software package developed at the University of Washington.  Right: Foregrounds in the Fourier domain, as observed by PAPER.  The EoR window is the blue region containing EoR signal but relatively little foreground contamination.  Current experiments have begun to set interesting power spectrum limits by working in the EoR window.  HERA will take full advantage of both views of the sky, using maps of smooth spectrum and polarized foregrounds to enlarge the EoR window.
%}\label{fig:twoFGViews}
%\end{figure}




\vspace{-0.25in}
\section{Project Description}
\label{PDsec}
\begin{figure}[!ht]\centering
\includegraphics[width=6.5in]{plots/PAPER_and_MWA.jpg}
\caption{\small
Reionization pathfinders PAPER (left) and MWA (right). Development of these instruments, now entering their final operational phases, provides the foundation for our understanding of foregrounds, sensitivity and instrumentation for building HERA.
}
\end{figure}
%\begin{figure}[!ht]\centering
%\includegraphics[height=1.75in]{plots/paper_element.jpg}
%\includegraphics[height=1.75in]{plots/hera_dish.png}
%\caption{\small
%The PAPER element (left) provides a clean instrumental response as a function
%of frequency \citep{parsons_et_al2010,parsons_et_al2012b}, which is crucial to
%the foreground isolation shown in Figure \ref{fig:eor_pspec}.  A 14~m dish
%designed around the same feed (right) dramatically improves sensitivity while
%constraining the path length and amplitude of
%reflections to ensure that foreground isolation is not substantially degraded.  
%}\label{fig:hera_dish}
%\end{figure}

This proposal outlines a staged build-out to a 568-antenna array in South Africa that incorporates the lessons learned from the first generation EoR observatories. It features a 14~m zenith-pointing dish optimized for sensitivity and spectral smoothness, a dense hexagonal core to enable redundant baseline calibration and delay-spectrum analysis, and a distribution of outrigger antennas to provide complete uv coverage to $\sim$700~m for foreground imaging and mitigation. HERA draws on the technical heritage of the MWA, PAPER, EDGES and MITEoR. Specific examples include the antenna feed and correlator of PAPER, receiver node and field digitization from the MWA, absolute radiometric calibration from EDGES, redundant baseline calibration from MITEoR, the delay-spectrum analysis from PAPER, and the precision imaging and foreground removal software from the MWA.

%Incorporated into the design of HERA are new aspects that reflect our current understanding of the optimal balance between sensitivity and foreground systematics.  
% XXX need to smooth out below... 
The HERA antenna is an example of this technical heritage. The spectral smoothness and the stability of the antenna response determine the precision to which astrophysical foreground emission can be separated from the cosmological 21~cm emission. HERA uses the PAPER dipole feed---modified slightly for wider bandwidth---suspended over a 14~m parabolic dish. The short ($\sim$5~m) focal height of the dishes is central to limiting the path length of reflections whose time-delay gives rise to chromatic antenna sensitivity. The zenith pointing enhances the stability of the antenna response (PAPER), short cables to in-field digitizers limit the length of cable reflections (MWA), and absolute calibration (EDGES) are all designed to provide an extremely stable and smooth spectral response. Similarly the antenna layout uses the dense core, outriggers, and symmetric configuration of the MWA, combined with redundant baselines within the core (PAPER, MITEoR). Together these advances enable HERA to have the science reach envisioned in the decadal survey while fitting within the MSIP funding envelope.


HERA follows a staged build-out plan.  In
each deployment stage improvements are incorporated into the system and new
science capabilities are unlocked.  This approach has the advantage of
providing early access to science and reducing the project risk by testing systems
early and changing them incrementally.  As shown in Figure \ref{fig:eor_pspec}, each
stage of HERA brings an associated improvement in sensitivity that allows key
aspects of 21~cm reionization science to be addressed.  The timeline of HERA
development, along with the associated science products, is outlined below. 


\noindent{\bf Year 1--Infrastructure and First 37 Antennas (FY 2015)}.  
\begin{itemize}\setlength{\parskip}{0pt}
\vspace{-7pt}
  \item Infrastructure installation. Using the new `K3' site approximately 10~km from the current PAPER site at the Karoo Radio Observatory in South Africa. Includes ground leveling, power and basic network connectivity.
  \item Move existing PAPER-128 antennas, correlator, and EMC container to K3 site.
  \item Install first 37 HERA antennas and instrument with existing PAPER feeds and electronics. 
  \item Begin development of wider bandwidth HERA baluns, receivers, feeds, and nodes using the PAPER and MWA techical heritage \citep{} and the development of in-situ antenna calibration system based on EDGES \citep{}. Continue development of delay-spectrum \citep{}, Fast Holographic Deconvolution software (FHD; \citealt{}) and automated redundant baseline calibration software \citep{Liu2010}.
\end{itemize}


\noindent{\bf Year 2--Hardware Commissioning and Deep Foreground Survey (FY 2016)}.  
\begin{itemize}\setlength{\parskip}{0pt}
\vspace{-7pt}
  \item Commissioning observations using a hybrid array of 37 HERA antennas in a close-packed hexagon surrounded by 91 PAPER antennas in an imaging configuration.
  \item Perform a deep polarized foreground survey using unique hybrid antenna capability of FHD. Will directly determine on-sky beam response of HERA antennas and enable future subtraction of sources in HERA sidelobes.
  \item Finalize site infrastructure (high-bandwidth optical network, surveying, trenching).
  \item On-antenna commissioning of new feeds, receivers, nodes, and antenna calibrations systems in Green Bank and South Africa.
  \item Build out to 127 HERA antennas starts.
\end{itemize}


\noindent{\bf Year 3--HERA 127 and Detecting the Rise and Fall of Reionization (FY 2017)}.
\begin{itemize}\setlength{\parskip}{0pt}
\vspace{-7pt}
  \item Construction of HERA~127 completes, and science observations begin in Oct.\ 2016, again using the PAPER correlator.
  \item Analysis begins on a dataset capable of constraining the timing and duration of reionization. Analysis focuses on current techniques based on PAPER individual baseline analysis, with exploration of subtracting bright and polarized foreground sources.
  \item Deployment of HERA~331 begins. Node electronics are installed for all 331 elements, and a new, 331-element GPU-based correlator is installed in the Karoo Array Processing Building (KAPB).
  \item Science papers from HERA~37 are published.
  \item  Additional data storage infrastructure is installed in the
KAPB.  The UPenn data analysis cluster is upgraded. 
\end{itemize}

\noindent{\bf Year 4--HERA 331 and Measuring the Evolution of the First Galaxies (FY 2018)}.
\begin{itemize}
\setlength{\parskip}{0pt}
\vspace{-7pt}
  \item Construction of HERA~331 completes, and science observations begin in Oct. 2017.
  \item Science observations with HERA~331 complete in Apr. 2018, and analysis begins on a dataset capable of
characterizing the redshift evolution the power spectrum shape---revealing the development of the first galaxies.
  \item Continued analysis and software development with an emphasis on opening the EoR window (removing contamination at low $k_{\parallel}$). 
  \item  HERA~127 results are published.
  \item Build out to 568 antennas begins, including outrigger antennas to fascilitate imaging and better foreground removal.
  \item Pipelines for EoR processing of HERA~568 complete.
\end{itemize}

\noindent{\bf Year 5--HERA 568, Imaging Reionization and Constraining the Development of Structure at $z>11$  (FY 2019)}.
\begin{itemize}
\setlength{\parskip}{0pt}
\vspace{-7pt}
  \item Construction of HERA~568 completes, and science observations begin in Oct. 2018.
  \item Analysis push to enable imaging of the largest structures and extracting the full sensitivity of the instrument (including partially coherent baselines). 
\end{itemize}


\subsection{The Transformative Science enabled by HERA}

% XXX Talk about specifically what science is being added in each stage of HERA
% XXX Bowman plot of constraints on ionization fraction versus redshift
% XXX new capabilities that are added in each stage
% XXX new low-frequency exploration, dark-ages plot from Aaron Ewall-Wice

In addition to pushing the sensitivity frontier, HERA will also extend the redshift frontier 
to $z \sim 20$ and possibly beyond.  Measuring the power spectrum at higher, pre-reionization redshifts provides
an incisive probe of astrophysical processes that are qualitatively different from those that
drive reionization.  During the reionization era, the $21\,\textrm{cm}$ signal is driven by
fluctuations in the ionization state of the IGM, while at higher redshifts the signal is determined
by fluctuations in the spin temperature.  These fluctuations in turn depend sensitively on the nature of early heating sources  and their abundance, as well as on more exotic physics such as dark matter annihilation cross-sections.  Beginning with the midstages of its staged build-out, HERA will be in a unique position to probe the pre-reionization epoch that current-generation instruments are unable to reach (see Figure \ref{fig:x_i_Xray}).  With its ability to measure the power spectrum \emph{continuously} over an extremely large frequency range,
HERA will also provide a longer lever arm for ionization history measurements than any other astronomical probe,
as one can see in Figure \ref{fig:x_i_Xray}.

{\sl Imaging reionization} In the later stages, HERA will become a powerful imaging instrument of cosmic reionization. Fiducial simulations of the expected HI 21cm signal on 25' scales predict regions with contrasts of about 10mK at 150MHz, or flux densities of up to 0.5 mJy/beam \citep{mcquinn_et_al2007}. The expected thermal noise for HERA 576 is 60 uJy, hence these regions could be detected at high confidence across redshifts $6<z<12$. Figure \ref{imaging} shows the predicted images of the HI 21 structures  assuming the properties of HERA 547. The large scale structure is easily detected, and imaged, with the later stages of HERA. Of course, reaching these sensitivity levels relies critically on foreground synchrotron removal. We will be exploring foreground removal techniques throughout the program, and in particular, in the latter stages. The resulting HI 21cm images will provide a key reference for imaging of the large scale galaxy distribution (the sources of reionization), using CO or [CII] intensity mapping and/or nearIR surveys with WFIRST (see section ??).

% TBD: include in full prop? no space in science pre-proposal
%\begin{figure}
%\centering
%\includegraphics[height=4.in]{plots/HERA_z8_SNR_annotated_v2.jpg}
%\caption{\small With a surface brightness sensitivity, after filtering chromatic foregrounds, of 60uJy, the completed HERA array has enough sensitivity to detect regions of HI and (in contrast) HII at SNR of 10 and above. Shown here is an instrumental simulation of observed HI emission assuming 100 hours of observing. Large shells of HI and deep HII regions are both detected at 10$\sigma$  \label{imaging}}
%\end{figure}

\section{Broader Impacts and Benefit to the Community}
\label{BIsec}

\subsection{Developing Young Instrumentalists (shorten)}

Aspiring instrumentalists in radio astronomy face a challenging path, in part because
the fundamental nature of radio astronomy instrumentation is changing.
Because the capabilities of radio telescopes are directly tied to the Moore's Law
growth in the capabilities of digital electronics,
new science opportunities are constantly being opened up, while existing
facilities rapidly fall into obsolescence.
Furthermore,
the breakneck pace of technological development places a premium on rapid development.
The faster an instrument incorporating new technology can be deployed,
the earlier that new science goals can be reached.

All of this combines to mean that aspiring radio astronomy instrumentalists must acquire a very broad
set of skills in a short amount of time.  A sampling of the skills required includes the practical
application of antenna design, optics,
3D modeling, machining, circuit design, soldering, DSP algorithms, FPGA/GPU programming, network management,
computing systems, kernel optimization, interferometry, synthesis imaging, astrometry, statistical methods, linear
algebra, high-performance
computing, laboratory methods, and many, many others.  While some of these skills are taught
in undergraduate and graduate
curricula, many are not, and exposure varies tremendously between majors.  In particular,
many students who arrive at graduate school in astronomy and physics
lack preparation in
practical engineering, programming, and laboratory skills.  Students with talent in physics, math, and astronomy,
but with a practical bent are often siphoned off in other directions.

One aspect of the broader impact of this proposal addresses the loss of talented
scientists with practical skills from applications in astronomy
instrumentation, and to address the gap between the preparation afforded by an
undergraduate education and the breadth of skills required to succeed as an
aspiring (radio) instrumentalist in graduate school.  This will be done by
funding a one-year junior specialist position in the RAL, offered annually over the course of this grant, to a
person to who will gain practical experience working alongside professional RAL engineers and scientists
on all aspects of the HERA instrumentation development at UC Berkeley, with the goal of
acquiring the practical skills required for
pursuing research in instrumentation in a graduate program.

\subsection{Developing African Scientists}

An important STEM activity of the HERA program in South Africa, and
one of the most personally fulfilling, is the cooperative education of
minority students from the third world. PAPER has an admirable history
of employing technical and engineering undergraduate and masters
students (formally known as 'interns'), from South African
Universities, notably, University of Kwazulu-Natal in Durban, during
major engineering deployments in South Africa. The work entails all
aspects of telescope, receiver, correlator, and infrastructure construction and
integration, and the work is incorporated into the UKNZ academic
program as a key 'practicum'. From the project perspective, this
student field engineering contribution has been fundamental for the
success of PAPER.

For HERA, we plan to expand this cooperative student support beyond
UKZN, and beyond interns. We have established formal collaborations
with faculty at UCT, UWC, and UKZN. Doctoral students from these
institutions will participate in all aspects of the project, from
field engineering to complex data analysis. We will establish PhD
student exchanges, where students from ZA and USA make extended visits
to HERA institutions in the partner Nation. Moreover, HERA will be
employing a number of PhD students throughout the project. We will
strongly encourage minority students from ZA to apply for these
PhD positions, in particular, those students that have come
through the intern program.

% TBD
%Notes:
%
%1. For the final proposal, we should get a supporting letter from 
%Durban and other Unviersities wrt student program. 
%
%2. I have read the 'Faculty Bridges' documents from Sheth at NRAO.
%Unfortunately, this program is just getting started, and it is not
%formally recognized by the NSF, nor funded. If this becomes more
%concrete over the next 4 months, we might consider adding a sentence
%that student exchanges will be facilitated by the formal agreements
%established by the 'Faculty Bridges' program.  However, for now, I
%have left this out.

\subsection{Benefits to the Community: Data Products}

We re-emphasize that observations of the HI 21cm line provide a truly unique probe of the evolution of the neutral gas content of the Universe during cosmic reionization, and into the preceding dark ages. In parallel, there are powerful techniques being explored to obtain an 'inverse view' of large scale structure during reionization, namely the distribution of the sources of reionization (star forming galaxies and AGN), and the ionized regions themselves.  The cross correlation between these two inverse views of large scale structure in the Universe will provide a complete imaging inventory of the sources and sinks of cosmic reionization, and fully constrain the physical processes involved in the formation of the first galaxies and AGN, and their influence on the neutral IGM (see Pritchard \& Loeb 2012 for a review).  Exploration of imaging of reionization on 20' scales is goal of the final 570 deployment of HERA, in year 5. This timescale is well matched to the expected first results of intensity mapping experiments of large scale structure in the galaxy distribution, as well as the final analysis of Planck  CMB anisitropies. 

CO and [CII] 'intensity mapping' experiments (1,2,3,4), and very wide-field near-IR galaxy surveys by WFIRST (5), are being design to trace-out the large scale galaxy distribution during reionization, corresponding to the sources driving reionization. Follow-up high resolution imaging of representative samples of these first galaxies with the JWST and ALMA will provide exquisite details of the distribution of the gas, dust, stars, and AGN in the first galaxies. Likewise, there are signatures of large scale structure in CMB images caused by streaming motions (the 'Doppler anisotropy') of the ionized structures during reionization (6, 7). These large scale ionized structures effectively 'fill-in' the holes seen in HI 21cm images of reionization. 

The cross correlation between very wide field galaxy distributions and the ionized gas distribution, with the HI 21cm results, increases the reliability of each measurement. In both cases, uncertainties due to systematic errors involved in removing foregrounds remain a dominant concern for the imaging results. However, these systematics are independent for the different techniques, and hence a cross correlation greatly mitigates the foreground contamination (4,6,7). We will work with teams from the complementary reionization studies to optimize the cross correlation analyses, thereby greatly enhancing scientific return on all experiments. 

% TBD:
%following could be throw-away/repeditive: 
%
%The combination of imaging of the neutral IGM from the 21cm experiments, with large scale galaxy distributions from CO/[CII] intensity mapping experiments and WFIRST surveys, and imaging of large scale ionized structures through CMB experiments, will map-out the full three dimensional structure of the Universe during reionization. In parallel, ALMA and JWST will probe the details of individual galaxies during reionization. Together, these programs will fulfill one of the primary Discovery objectives of NWNH, namely, exploring cosmic dawn and the epoch of first galaxy formation.
%
%1. A. Liu and M. Tegmark. PRD, 83:103006, 2011.
%2. C. L. Carilli, ApJ, 730:L30, 2011.
%3. A. Lidz et al. ApJ, 741:70, 2011.
%4. Y. Gong et al. ApJ, 728:L46, February 2011.
%5. http://www.ipac.caltech.edu/wfirst/overview/science/surveys/
%6. Alvarez et al. 2006, ApJ 647, 840
%7. Tashiro et al. 2010, MNRAS 420, 2617
%
%
%notes:
%
%1. Maybe Steve F. can put in some more interesting detailed physical implications?
%
%2. James can fill-in some CMB pol cross correlation details
%
%2. perhaps include fig 18 from Pritchard and Loeb (from Lidz et al., I think), or maybe one of the cross correlation PS analyses?

(MIT) Data products disseminated at MIT.  Products include:
integrated data products, made available after 2 years;
high-speed continuum images, generated as part of operational data analysis by MIT;
wide-field sky maps (made with 91 PAPER antennas in first 2 years),
database of metadata  for asscoiated with data products,
access to calibrated, compressed visibilities upon request.
deep (foreground subtracted) image products for use with other surveys \& instruments

\section{Project Management Plan}

This project balances the light-weight
management structure of PAPER/MWA activities and the more formal structure
required for larger-scale projects.  Building on PAPER's excellent track record
and the resources of UC Berkeley's Radio Astronomy Laboratory (RAL),
this proposal consolidates responsibility and management of HERA
fabrication and construction activities at RAL. Parsons serves as
Project Director/Scientist; DeBoer serves as Project Manager and design engineer.
Executive
(Aguirre, Bradley, Hewitt, Morales, Werthimer)
and Scientific (Aguirre, Bowman, Carilli, Furlanetto, Hewitt, Morales, Tegmark) Advisory Boards advise Parsons.
A Site Manager supervises and manages
construction activities executed by local contractors at the South African
site, and reports to DeBoer.  A SKA-SA Liaison (supported by SKA-SA)
interacts with DeBoer, the Site Manager and the SKA-SA board to
coordinate HERA, Meerkat, and SKA site activities.

An external advisory board evaluates the baseline design
based on existing components in a Preliminary Design Review in Year 1.
A Critical Design Review in Year 2 evaluates the
production design that incorporates improvements to the analog system
and feeds (Bradley), the node and correlator (Werthimer), the data storage
system (Aguirre), and analysis software (Morales).
Subsystem fabrication (electronics, antenna sub-assemblies, antenna assembly)
are contracted to industry partners in the US and South Africa.   Construction on site proceeds
with 2-3 teams of local contractors and laborers, managed by the site manager.  Observing proceeds 
autonomously without local observers; SKA-SA
provides occasional site support as necessary.

The budget is contingency-weighted at a low ($\sim$15\%) level, anticipating that project 
risk and contingency are handled primarily by de-scoping along the axes of scale, time, and new capability.  
A baseline design using
existing hardware establishes a low-risk path to core functionality and science.  Improved functionality
results from successful development activities or is otherwise de-scoped.
Cost excesses and schedule slips are absorbed by reducing
build-out, with associated de-scoping of science capabilities.

%\begin{table}
%\label{tab:params}
%\begin{tabular}{|l|rl|l|}
%\hline
%Parameter & Value & Units & Description \\
%\hline
%$N$  &  576  & & Number of Antennas \\
%$d$ & 14 & m & Antenna Diameter \\
%$f/d$ & 0.32 &  & Focal Length (fractional) \\
%$\Omega_{\rm P}$ & 0.026 & sr & Field of View (power) \\
%$\Omega_{\rm PP}$ & 0.013 & sr & Field of View (power$^2$) \\
%$\Omega_{\rm eff}$ & 0.052 & sr & Field of View (sensitivity) \\
%$B_{\rm samp}$ & 0--250 & MHz & Sampled Frequency Range \\
%$B_{\rm corr}$ & 100 & MHz & Correlated Bandwidth \\
%Config.    & 24 $\times$ 24 & & Square Grid Antenna Configuration\\
%$f/f_0$ & $1.5\cdot10^5$ & & Redundancy Metric (Parsons et al. 2012a) \\
%$A$ & 0.09 & km$^2$ & Total Collecting Area \\
%$\theta$ & 15 & arcmin & Angular Resolution (150 MHz) \\
%$b_{\rm max}$ & 500 & m & Maximum Baseline \\
%$T_{\rm sys}$ & 500 & K & System Temperature \\
%$t_{\rm obs}$ & 120 & days & Observing Time \\
%$t_{\rm day}$ & 6 & hrs & Observing Time Per Day\\
%$\Delta_{\rm N}^2$ & 1.6 & mK$^2$ & Expected Noise Level ($k=0.2 h\ {\rm Mpc}^{-1}$) \\
%SNR$_{21}$ & 11.7$\sigma$ &  & Expected Detection Significance (Lidz et al. 2008, $x_i=0.5$, 150 MHz) \\
%\hline
%\end{tabular}
%\end{table}

\clearpage
\setcounter{page}{1}
\thispagestyle{empty}
%\bibliographystyle{apj}
%\bibliographystyle{hapj}
\bibliographystyle{jponew}
%\bibliographystyle{unsrt}
\bibliography{biblio}


\end{document}

