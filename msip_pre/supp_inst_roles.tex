\documentclass[preprint]{aastex} 

\usepackage[top=1in, bottom=1in, left=1in, right=1in]{geometry}
\setlength{\parskip}{0pt}
\setlength{\parsep}{0pt}
\setlength{\headsep}{0pt}
\setlength{\topskip}{0pt}
\setlength{\topmargin}{0pt}
\setlength{\topsep}{0pt}
\setlength{\partopsep}{0pt}
\setlength{\footnotesep}{8pt}
\pagestyle{empty}

\begin{document}
\title{Supplementary Document:  Partner Institutions and Roles}

\noindent
Listed below are the partner institutions that are to be funded
via subawards, along with the role of each in the project.
\vspace{0.25in}
%XXX mention shared project roles here: construction of HERA-37,


\section*{University of California, Berkeley (UCB)}

UCB, building on Parsons' and Werthimer's expertise in digital
instrumentation for correlator development for PAPER/LEDA
and DeBoer's expertise in antenna development for the Allen Telescope
Array, will develop the node, correlator, and dish design for HERA,
drawing on the resources of several engineers and technicians in
the UCB Radio Astronomy Laboratory.  
These efforts will be coordinated and supervised by Werthimer.  
UCB engineers and staff, together will a graduate student focused
on instrumentation, will perform hardware commissioning, and will
perform data analysis for characterizing antenna and system performance
on the basis of the delay-space metrics that are central to
controlling foregrounds.

UCB will be responsible for issuing contracts for the fabrication,
shipment, and deployment of the HERA instrument, including the antennas,
analog electronics, nodes, the correlator, and infrastructure.
DeBoer 
will oversee the design, construction, and operational phases of the
project, coordinating efforts with the site team in South Africa leading
infrastructure development, the leaders of the fabrication, 
shipment, and deployment contracts, 
the site manager and Project Engineer, Goeke, and the leads of
the analog, digital, data, and software subsystems.

Parsons will lead two graduate students and a postdoctoral researcher
in continuing to develop and apply software for power-spectrum data
analysis based on the delay-spectrum and covariance-diagonalization
techniques that have been successfully applied on PAPER data.  

\section*{University of Pennsylvania (UPenn)}

The UPenn group will host the central computing
cluster and data archive for HERA. Penn currently performs this function for
PAPER. The cluster will be maintained by an IT professional. Penn will also
manage the flow of data flow from South Africa to the central storage, and
thence to collaborators as necessary.  In addition, building on experience
with the PAPER data compression, UPenn will provide the quality
assurance (QA) of raw data, including both software and hardware checking and
reporting on the data in real time. This will be the responsibility of the
postdoctoral fellow and the graduate student, with the goal of making science
quality data available as soon as possible after observation.  
Finally, UPenn will continue efforts begun on PAPER
investigating polarization effects on measured reionization power spectra,
providing feedback on the performance of the antenna design, providing
polarization calibration, and pioneering polarization leakage mitigation
techniques as necessary. This, combined with power spectrum analysis, will form
the basis for a graduate student thesis.

\section*{Massachusetts Institute of Technology (MIT)}

Hewitt and Tegmark, with two graduate students and a postdoctoral
researcher, will be responsible for 
operational data analysis, performing routine imaging and calibration
for the purpose of evaluating instrument performance, and generating
and distributing data products to the broader community outside of HERA.
Additionally they will develop and apply optimal estimator techniques
for power spectrum analysis with HERA data, and will undertake analysis
of low-frequency (<100 MHz) HERA data to characterize foregrounds and
perform the Dark Ages science component of this proposal.

Goeke, building on experience as a System Engineer for the
MWA, will
ensure that site construction proceeds on schedule and according to plan,
and will engage with the antenna subcontractor hired by UCB to
optimize the antenna design for manufacturability.

%transient science
%cross-correlation with other data sets

\section*{University of Washington (UW)}

Morales, with a graduate student and a postdoctoral researcher,
will develop the Monitor \& Control system
for HERA that s responsible for normal telescope operation and collection of the
meta-data needed for precision cosmology data analysis. They will also be
responsible for imaging using the Fast Holographic Deconvolution software. This
will include image-domain analysis of foregrounds with HERA 37 and developing the foreground
measurements needed to enhance the sensitivity of HERA 127, 331, and 568. UW
will also play an active role in the commissioning of the instruments and
analysis of all data using expertise in precision cosmology observations from the MWA.

\section*{National Radio Astronomy Observatory (NRAO)}

Using the PAPER analog subsystem as a baseline, NRAO (Bradley, assisted by a
technician) will develop the RF electronics design and optimize the feed design
to maximize the illumination of the parabolic dish and minimize the beam
ellipticity that gives rise to raw polarization leakage of the instrument.
These systems will be installed and evaluated on two two prototype parabolic
reflectors in Green Bank whose construction is provided for in the UC Berkeley
budget.  The optimized and tested design will be furnished to UC Berkeley for
fabrication.

In addition, NRAO scientists (Carilli, assisted by a Cambridge graduate
student) will perform data analysis for foreground and 21cm reionization
science, investigating flagging, calibration, and imaging using the CASA
software, and collaborating with the related efforts of other HERA partners.
NRAO will develop CASA software with enhancements for imaging with wide-field,
wide-band data, and will support analysis efforts by handing debugging,
instruction, and enhancement requests.

\section*{Arizona State University (ASU)}

Develop and apply In-situ beam calibration systems (octocopter?) and develop
some absolutely-calibrated balun/receivers.   Application of higher-order
statistics to 21cm reionization science.  science commissioning total-power
balun development octocopter beam calibration data calibration


\section*{University of California, Los Angeles (UCLA)}

At UCLA, Furlanetto will develop tools for the final stages of data analysis in the HERA
project, building tools to bridge the gap between the data products and the
physics questions of galaxy formation and evolution. Specifically, this work
entials updating Furlanetto’s two existing semi-numerical codes
(the public codes 21cmFAST and DexM) to match the parameters and 
strategy of the HERA experiment, and then
simulating sources of the spin-flip background, identifying and
varying the key parameters that HERA measurements will constrain.
Experimental data will then be compared to these simulations in order to
extract quantitative constraints on these parameters. This work will be done by
Furlanetto and a graduate student, for whom this project will constitute their
Ph.D. thesis.

\end{document}
