\documentclass[preprint]{aastex}
\usepackage[top=1in, bottom=1in, left=1in, right=1in]{geometry}
\usepackage{amsmath}
\usepackage{graphicx}
\usepackage{mdwlist}
\usepackage{natbib}
\usepackage{natbibspacing}
\setlength{\bibspacing}{0pt}
\setlength{\parskip}{0pt}
\setlength{\parsep}{0pt}
\setlength{\headsep}{0pt}
\setlength{\topskip}{0pt}
\setlength{\topmargin}{0pt}
\setlength{\topsep}{0pt}
\setlength{\partopsep}{0pt}
\setlength{\footnotesep}{8pt}
\pagestyle{empty}

%Project Summary. (1 page maximum) Required elements include an overview of the
%proposed program, and separate entries addressing the intellectual merit and
%broader impacts. The summary should be written in the third person, informative
%to those working in the same or related field(s), and understandable to a
%scientifically or technically literate reader.

\begin{document}
\pagestyle{empty}

\title{HERA: Illuminating Our Early Universe}

\section*{Overview}
%Overview: Insert a self-contained description of the activity that would result if the proposal were funded and include a statement of objectives and methods to be employed. 

As endorsed in the recent astronomy decadal survey, 
Hydrogen Epoch of Reionization Arrays (HERA) is a roadmap for characterizing
cosmic reionization --- the epoch when the first luminous sources ionized the
bulk of the hydrogen in the Universe --- via redshifted 21-cm hyperfine
emission from neutral hydrogen.  Following on the successes of first-generation
HERA instruments (PAPER, MWA, LEDA, MITEoR, and EDGES), we propose to build
the next phase of HERA using 14-m fixed-pointing parabolic dishes, deployed in
stages of 127, 331, and 568 elements, observing from 50 to 225 MHz.
This instrument brings to bear both experimentally validated
foreground suppression techniques and the collecting area required to open new windows into our early
universe.  

Each stage of HERA delivers new science capabilities:
HERA 127 will be capable of determining
ionization fraction versus redshift over the bulk of reionization,
HERA 331 will constrain the size of ionization bubbles to help determine the properties of
first galaxies, and HERA 568 directly image larger structures during
reionization.  HERA will also begin to explore the Dark Ages before 
reionization, when a wealth of astrophysical and cosmological processes
impacted the temperature of the intergalactic medium.

\section*{Intellectual Merit}
%Intellectual Merit: Describe the potential of the proposed activity to meet the Intellectual Merit criterion

The 21-cm hyperfine transition is widely recognized as potentially the most
powerful probe of large-scale structure from the dark ages through
reionization.  The evolution of this signal from the neutral intergalactic
medium depends on myriad physical processes; the relative importance of
these competing effects is sensitive to, among other things, the expansion of
the universe, the ignition of the first stars and galaxies, the formation of
the first massive black holes, and the relative
velocity of baryonic matter and dark-matter halos.
Detecting this signal will have an impact
comparable to the discovery of the cosmic microwave background, and study of the three-dimensional
evolution of large-scale structure via the 21-cm line has the
potential to become `the richest of all cosmological data sets'.

Current constraints on cosmic reionization remain rudimentary. When did it
occur, and over what timescale?  What objects dominated the radiation field?
How were the objects distributed? 
What were the most important feedback mechanisms in the transition
from the first stars to first galaxies, and how did they affect these populations?
Without new constraints, further
progress on theoretical modeling of first galaxy formation and cosmic
reionization remains problematic.  
HERA provides the key measurements
that are needed to
advance our understanding of early galaxy formation and
cosmic reionization.
The proposed HERA program provides a
powerful new capabilities for producing observational constraints on reionization and
large-scale structure evolution during the dark ages.  The new
windows that HERA opens into our early universe have the capability to
transform our scientific understanding of the complex interaction of cosmology
and astrophysics during our cosmic dawn.  HERA represents a major step toward unlocking the widely recognized scientific potential of
21-cm cosmology.

\section*{Broader Impacts}
%Broader Impacts: Describe the potential of the proposed activity to meet the Broader Impacts criterion

The HERA program will train new
instrumentalists at the graduate and undergraduate levels, increase the
diversity of US graduate programs by engaging South African students and
preparing them for admission to US degree programs, and make our major data
products available publicly.
HERA will involve graduate students in all
stages of HERA development and observation. We also fund an
undergraduate specialist position
at UC Berkeley's RAL, offered annually,
mentoring the individual in the skills required to pursue graduate
research in instrumentation.

Another important HERA activity is the cooperative education of
under-served students from South Africa in STEM fields.
We have established formal collaborations
with faculty at South African universities to engage doctoral students in the HERA project.
We will establish
student exchanges to enhance
the diversity of US graduate programs by preparing South African students for
admission to US degree programs.
Involvement in the
program will help ensure that these students are well-positioned for
future success.

Finally, HERA will disseminate various data products to the community,
including wide-field sky maps, high-speed continuum images, deep
foreground-subtracted images, and full-sensitivity data products,  These enable
auxiliary science, including cross-correlation analyses with other 
complementary large-scale
probes of reionization.

\end{document}
