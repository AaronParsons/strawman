\section{The HERA Road Map}
The Hydrogen Epoch of Reionization Array (HERA) is a scientific road
map aimed at exploring the large-scale structure in the baryonic
universe via the 21cm line of hydrogen.  A white paper describing this
investigation was submitted to the U.S. National Research Council's
2010 Committee for the Decadal Survey of Astronomy and Astrophysics,
Program Panel: Radio, Millimeter, and Submillimeter (RMS).  A
three-phased program was proposed: I) the detection of the power
spectrum of the 21 cm line emission structures; II) the
characterization of the structures including galaxy formation
astrophysics and cosmological physics; and III) the detailed imaging
of structures across a large fractional step in cosmic time. The
science, which was highly ranked by the Committee, is aimed toward
answering such major questions as what objects first lit up the
Universe? When did this occur? How has the Universe evolved over time?
In addition, the Committee recommended the creation of the Mid-Scale
Innovations Program at the National Science Foundation (NSF) to fund
projects such as HERA.

The current HERA I initiatives, PAPER, the Murchison Widefield Array
(MWA), and the Experiment to Detect the Global EoR Step (EDGES), are
expected to culminate in a detection and will define the scope of the
HERA II characterization milestones.  The imaging task of HERA III
will require a very large-scale array to achieve the necessary
sensitivity over a wide range of spatial frequencies. It is expected
that this will be achieved by the Square Kilometer Array (SKA) in the
coming decade and will incorporate into its design the accrued body of
knowledge gained from HERA I and II activities.  Hence, the
accumulation of experience with each step in the HERA program along
with parallel technical development are required to achieve the goals
of signal detection, detailed statistical characterization, and
structural imaging.

The national observatories, in partnerships with universities, are
important resources in the HERA road map paradigm.  At the beginning
of an investigation, the small scale of the instrumentation is such
that it can be managed effectively as a PI driven activity. However,
the observatories can assist the PI with technical developments and
provide opportunities for deploying the instrument in an
RFI-controlled environment near key infrastructure needs such as
power, Internet, housing, technical expertise, works area assistance,
etc.  As the instrument grows in both complexity and size, the
observatory can help manage the construction and evaluation of the
hundreds of components needed for the array projects.  It is an
opportunity for students from the universities to work with
observatory staff to learn project development and management skills
without being overwhelmed by the construction effort itself.  Upon
completion of the project, the operation, data management, and data
analysis tasks can be handled by the students and researchers at the
universities.  These partnership can be optimized, on a case-by-case
basis, to help achieve the milestones of the scientific investigation
while still maintaining the PI/Co-I driven approach to HERA.

In 2011, the HERA Coordinating Committee was formed to bring together
representatives from each of the current HERA I activities to share
lessons learned and begin charting a course toward the next milestone.
The Committee consists of three members of the MWA team: J. Hewitt
(MIT), M. Morales (U. Washington), and J. Bowman (Arizona State), and
three members of the PAPER team: R. Bradley (NRAO / U. Virginia),
A. Parsons (U.C. Berkeley) and C. Carilli (NRAO).  The EDGES project
was also represented by J. Bowman. Through a series of monthly
telecons, the HERA Coordinating Committee members exchanged detailed
information about the projects and discussed the various technologies
and methodologies that were utilized.  The work of the committee over
the past year has culminated in the writing of this paper.

The purpose of this paper is to introduce the HERA road map to the
astronomical community and provide details of its implementation. Part
II gives a brief topical overview of the science, a snapshot of our
current understanding of EoR in the context of 21cm cosmology, with
emphasis placed on those empirical and theoretical results that impact
the boundaries of new instrument designs.  Our acquired body of
knowledge is summarized in Part III, where we provide an overview of
the various HERA I projects and a synopsis of our methodological
lessons learned.  Finally, in Part IV, we extrapolate our current
findings into the future by providing the motivational framework and
conceptual design for achieving the next major milestone array along
the HERA trail.
